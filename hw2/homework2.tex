\documentclass{report}

\usepackage[a4paper]{geometry}

\usepackage{libertineotf}
\setmonofont{Inconsolata}
\newfontfamily{\cyrillicfonttt}{Inconsolata}

\usepackage{xecyr}
\usepackage{polyglossia}
\setmainlanguage{bulgarian}

\usepackage{amsmath}
\newtheorem{pr}{Задача}
\newcommand\sol{\section*{Решение} }

\usepackage{csvsimple}
\usepackage{wrapfig}
\usepackage{enumitem}

\usepackage[noae]{Sweave}
\begin{document}

\title{Домашно упражнение №2}
\author{Петко Стоянов Борджуков\\\small{Ф№ XXXXXX, курс XX, група XX}}

\maketitle

\begin{pr}
  Напишете функция в средата R, която за произволно $n$ прави $m$ симулации на
  броя неподвижни точки на пермутациите на числата от $1$ до $n$. Функцията да
  връща оценки за средното и дисперсията на случайната величина брой неподвижни
  точки при пермутация на числата от $1$ до $n$. Приложете я за $n = 10, 20, 30$
  и $m = 1000$. Резултатите близки лиса до теоретичните?
\end{pr}

\sol
Нека дефинираме функцията \texttt{standing\_points}, която приема параметри $n$ и $m$ със стойност по подразбиране $1000$ така:
\begin{Schunk}
\begin{Sinput}
> standing_points = function(n, m=1000) {
+   results = numeric(0)
+   for(i in 1:m) {
+     count = 0
+     permutation = sample(1:n, n)
+ 
+     for(j in 1:n){
+       if(permutation[j] == j) count = count + 1
+     }
+ 
+     results[i] = count
+   }
+   c(mean(results), var(results))
+ }
\end{Sinput}
\end{Schunk}

Нека изпълним функцията за $n = 10, 20, 30$:
\begin{Schunk}
\begin{Sinput}
> standing_points(10)
\end{Sinput}
\begin{Soutput}
[1] 1.011000 1.057937
\end{Soutput}
\begin{Sinput}
> standing_points(20)
\end{Sinput}
\begin{Soutput}
[1] 1.0170000 0.9516627
\end{Soutput}
\begin{Sinput}
> standing_points(30)
\end{Sinput}
\begin{Soutput}
[1] 1.027000 1.063334
\end{Soutput}
\end{Schunk}

Резултатите са близки до теоретичните.

\begin{pr}
  Да се направи симулация на централна гранична теорема, когато имаме микс от
  две разпределения, за които очакването на сл. величини от тези разпределения е
  $1$ и дисперсията също е $1$. Разпределенията са по Ваш избор, както и
  пропорцията на сл. вел. от двете разпределения.
\end{pr}

\sol
Нека дефинираме функцията \texttt{clt}, която приема за аргументи брой
симулации, $n$ – брой наблюдения при симулация, $\mu$ – средна стойност и
$\sigma$ – стандартно отклонение. Нека тази функция да съставя списък от
случайни величини. Тези случайни величини са получени в съотношение $1:1$ от
нормално и експоненциално разпределение с очакване $\mu$ и дисперсия $\sigma$.

\begin{Schunk}
\begin{Sinput}
> clt = function(simulations=100, n=100, mu=1, sigma=1) {
+   results = c()
+   for(i in 1:simulations) {
+     X = rnorm(n, mu, sigma)
+     Y = rexp(n,1/mu)
+     results[i] = (mean(X) - mu)/(sigma/sqrt(n))
+     results[simulations+i] = (mean(Y) - mu)/(mu/sqrt(n))
+   }
+   results
+ }
\end{Sinput}
\end{Schunk}

Нека извършим $5000$ симулации с параметри $\mu = 1$ и $\sigma = 1$ и да сравним получената хистограма с кривата на $N(0, 1)$:
\begin{center}
\begin{Schunk}
\begin{Sinput}
> hist(clt(5000), prob=T)
> curve(dnorm(x), add=T)
\end{Sinput}
\end{Schunk}
\includegraphics{homework2-004}
\end{center}
Данните ни са приблизително нормално разпределени.

\begin{pr}\label{pr:3}
  Решете:

  \begin{enumerate}[label=(\alph*)]
  \item Нека $\xi \sim N(0,1)$. Намерете $z^\ast$, такова че $P(-z^\ast < \xi <
    z^\ast) = 0.6$.
  \item Нека $\eta \sim Ge(0.3)$. Намерете $P(1 \leq \eta \leq 4)$.
  \item Нека $\zeta \sim Exp(1)$. Намерете $q$, такова че $P(\zeta <
    q) = 0.5$.
  \end{enumerate}
\end{pr}

\sol
\begin{enumerate}[label=(\alph*)]
\item Вероятността случайната величина да е в интервал $(-z^\ast; z^\ast)$ e
  лицето, заградено от плътността на нормалното разпределение между $-z^\ast$ и
  $z^\ast$. Понеже средното на това разпределение е $0$, плътността е четна
  функция и лицето между $-z^\ast$ и $0$ е равно на лицето между $0$ и $z^\ast$
  – половината от цялото лице. Тогава $z^\ast$ се изчислява така:
\begin{Schunk}
\begin{Sinput}
> z = abs(qnorm(0.5 - 0.3))
\end{Sinput}
\end{Schunk}

  Проверка:
\begin{Schunk}
\begin{Sinput}
> pnorm(z)-pnorm(-z)
\end{Sinput}
\begin{Soutput}
[1] 0.6
\end{Soutput}
\end{Schunk}
\item Вероятността $P(1 \leq \eta \leq 4)$ се пресмята така:
\begin{Schunk}
\begin{Sinput}
> pgeom(4, 0.3) - pgeom(1, 0.3)
\end{Sinput}
\begin{Soutput}
[1] 0.32193
\end{Soutput}
\end{Schunk}

\item Стойността на $q$ изчисляваме така:
\begin{Schunk}
\begin{Sinput}
> qexp(0.5, 1)
\end{Sinput}
\begin{Soutput}
[1] 0.6931472
\end{Soutput}
\end{Schunk}
\end{enumerate}

\begin{pr}
  За $\eta$ от задача~\ref{pr:3} да се начертае дискретната плътност
  (вероятностното разпределение: $k$ срещу $P(\eta = k)$). На същата графика, но
  с различни символи, да се добавят разпределенията на $\eta_1 \sim Ge(0.25)$ и
  $\eta_2 \sim Ge(0.4)$.
\end{pr}

\sol

\begin{center}
\begin{Schunk}
\begin{Sinput}
> xvals = 0:30
> hist(rgeom(1000, 0.3), prob=T)
> points(xvals, dgeom(xvals, 0.3) , type='p', pch='o', lwd=3, col='green')
> points(xvals, dgeom(xvals, 0.25), type='p', pch='x', lwd=2, col='blue')
> points(xvals, dgeom(xvals, 0.4) , type='p', pch='+', lwd=2, col='red')
> legend(c("topright"), c("0.25","0.3", "0.4"),
+        pch=c('x', 'o', '+'),col=c("blue","green","red"))
\end{Sinput}
\end{Schunk}
\includegraphics{homework2-009}
\end{center}
\end{document}
