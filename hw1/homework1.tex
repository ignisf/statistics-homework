\documentclass[draft,12pt]{report}

\usepackage[a4paper]{geometry}

\usepackage{xecyr}
\usepackage{polyglossia}
\setmainlanguage{bulgarian}

\usepackage{libertineotf}

\usepackage{amsmath}
\newtheorem{pr}{Задача}
\newcommand\sol{\emph{Решение}. }

\begin{document}

\title{Домашно упражнение №1}
\author{Петко Стоянов Борджуков\\\small{Ф№ 61322, курс IV, група IV}}

\maketitle

\begin{pr}
Изследвали сте между 25 и 30 индивида за целите на някакво проучване. Следили
сте за три показателя на обектите на изследването -- два качествени и един
количествен или един качествен и два количествени. Опишете структурата от данни
чрез data frame в средата R. Направете дескриптивни статистики за всяка
характеристика и интерпретирайте. Представете графичнo трите променливи
поотделно, също така и по двойки за тези от тях, за които има смисъл и за които
бихте предположили някакви зависимости или отсъствие на такиав. Изведете
номера/номерата на обектите, които имат максимални стойности за числовите
характеристики. Изведете целия ред от наблюдения за тези индивиди.
\end{pr}

\end{document}